%----------------------------------------------------------------------------
%\chapter{Rövidítések és fordítások}
\chapter*{Rövidítések és fordítások}
\label{sec:abbreviations}
\addcontentsline{toc}{chapter}{Rövidítések és fordítások}
%----------------------------------------------------------------------------

A szakterület rövidítései
t és az előforduló angol kifejezéseit lehetőségemhez mérten  próbáltam magyarítani illetve feloldani az első előfordulásukkor.
Ahol találtam már előforduló magyar kifejezést, azt használtam, azonban az esetek nagy részében ez nem állt fent, így elnézést az esetlegesen váratlan és meglepő kifejezésekért.
Emiatt és a könyebb kereshetőség érdekében összegyűjtöttem a rövidítéseket és az általam használt magyarításokat.


\begin{multicols}{2}
\textbf{QPS} Queries per Second 

\textbf{HPA} Horizontal Pod Autoscaler / Automatikus Horizontális Pod Skálázó 

\textbf{VPA} Vertical Pod Autoscaler / Automatikus Vertikális Pod Skálázó 

\textbf{CPU} Processzor 

\textbf{I/O} Input and Output / Kimenet és Bemenet 

\textbf{K8s} Kubernetes 

\textbf{CNCF} Cloud Native Computing Foundation 

\textbf{Control plane} Vezérlő sík 

\textbf{Node} Csomópont 

\textbf{Pod} Kapszula 

\textbf{Kube Scheduler} Ütemező 

\textbf{CR} Custom Resource / Saját Erőforrás 

\textbf{CRD} Custom Resource Definition / Saját Erőforrás Leíró

\textbf{SLA} Service-level Agreement / szolgáltatási szint megállapodásokba

\textbf{QoE} Quality of Experience / érzékelhető szolgáltatási szint

\columnbreak

\textbf{API} Application Programming Interface / Alkalmazásprogramozási felület 

\textbf{OLM} Operator Lifecycle Manager / Operátor Életciklus kezelő 

\textbf{FE} Front-end 

\textbf{BE} Back-end 

\textbf{HTTP} Hypertext Transfer Protocol 

\textbf{HTML} Hypertext Transfer Protocol %TODO

\textbf{Tight loop} Szoros ciklus 

\textbf{kubectl} Kubernetes parancssoros kezelője 

\textbf{Liveness probe} Életteli próba 

\textbf{Startup probe} Indítási próba 

\textbf{Readiness probe} Készenléti próba 

\textbf{Service Mesh} Szolgáltatás háló

\textbf{Black Box} Fekete doboz

\textbf{White Box} Fehér doboz


\end{multicols}

%Todo
