%----------------------------------------------------------------------------
\chapter{Rövidítések és fordítások}
\label{sec:abbreviations}
%----------------------------------------------------------------------------

A szakterület sok rövidítést és mégtöbb angol kifejezést tartalmaz, amiket a lehetőségemhez mérten próbáltam magyarítani a könyebb olvashatóság érdekébe, illetve feloldani az első előfordulásakor.
Ahol találtam már előforduló magyar kifejezést, azt használtam, azonban az esetek nagy részében ez nem állt fent, így elnézést az esetlegesen váratlan és meglepő kifejezésekért.
Emiatt és a könyebb kereshetőség érdekében összegyűjtöttem a rövidítéseket és az általam használt magyarításokat.

\paragraph{QPS} Queries per Second
\paragraph{HPA} Horizontal Pod Autoscaler / Automatikus Horizontális Pod Skálázó
\paragraph{VPA} Vertical Pod Autoscaler / Automatikus Vertikális Pod Skálázó
\paragraph{CPU} Processzor
\paragraph{I/O} Input and Output / Kimenet és Bemenet
\paragraph{K8s} Kubernetes
\paragraph{CNCF} Cloud Native Computing Foundation
\paragraph{control plane} vezérlő sík
\paragraph{node} Csomópont
\paragraph{Pod} Kapszula
\paragraph{Kube scheduler} Ütemező
\paragraph{CR} Custom Resource / Saját Erőforrás
\paragraph{CRD} Custom Resource Definition / Saját Erőforrás Leíró
\paragraph{SLA} Service-level Agreement / szolgáltatási szint megállapodásokban
\paragraph{QoE} Quality of Experience / érzékelhető szolgáltatási szint
\paragraph{API} Application Programming Interface / Alkalmazásprogramozási felület
\paragraph{OLM} Operator Lifecycle Manager / Operátor Életciklus kezelő
\paragraph{FE} Front-end
\paragraph{BE} Back-end
\paragraph{HTTP} Hypertext Transfer Protocol
\paragraph{Tight loop} Szoros ciklus
\paragraph{kubectl} Kubernetes parancssoros kezelője
\paragraph{Liveness probe} Életteli próba
\paragraph{Startup probe} Indítási próba
\paragraph{Readiness probe} Készenléti próba
%Todo