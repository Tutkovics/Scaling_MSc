%----------------------------------------------------------------------------
\chapter{Összefoglalás}
\label{sec:summary}
%----------------------------------------------------------------------------


%----------------------------------------------------------------------------
\section{Elvégzett munka}
%----------------------------------------------------------------------------
%Alapvetően jól működik a beépített skálázó.
%Pár extra lépéssel javítható lenne.
%Vagy nagyon komplex lehetőség vezetne csak sikerre.

Szeretném röviden összefoglalni a diplomamunkában tárgyalásra kerülteket és ezáltal az elvégzett munkát is, hogy a korábban végigvezetett gondolatmenet részei egészben is megjelenjenek.

A fő vizsgálandó terület a terhelésfüggő HPA skálázó működése volt szolgáltatáshálókon.
Ez a kérdés egyre több ágazatot érint, ahogy egyre többen választják az alkalmazásaik futtatására a felhőt, mint platformot.
Bevezetésként felvázoltam ezen folyamatot, annak beazonosítható elemeit és a jelenlegi tendenciákat.

Ezután a technikai részletek és megvalósítás előtt bemutattam a diplomadolgozat központi szerepét játszó Kubernetes klaszter felépítést, és a benne elérhető erőforrások szabályozását és a beépített skálázóját.

Az érdemi feladatok előtt fel kellett mérni a szakmán belül elérhető korábbi kutatásokat. Ismertettem a főbb elméleti vonalakat és ötleteket gyűjtöttem a saját munkához.

A kérdések megválaszolásához sok mérést kellett végezni, melyhez nélkülözhetetlen volt implementálni egy erre alkalmas környezetet. 
A környezet több különálló részből állt.
Létre kellett hozni egy saját erőforrást a Kubernetesen belül és implementálni egy kezelő logikát hozzá.
Szükség volt egy tetszőlegesen konfigurálható alkalmazásra, ami előre beállított végpontokra érkező kéréseket szolgál ki.
A mérés vezénylését és a kapott eredmények megjelenítését egy-egy külön alklmazás végzi.

Méréseken keresztül először megismertük, hogyan viselkednek a szolgáltatáshálók forgalomszabályozás nélkül egy túlterhelt rendszerben. 
Láttuk, hogy egy átbukási pont után a kiszolgálási paraméterek nagymértékben visszaesnek.

Következő kérdés, amire válasz született, hogy ezeket az eseteket mennyire csillapítja a HPA.
Láthattuk, hogy szűkös erőforrások esetén nem minden esetben sikerül elérni az elméleti optimum pontot, és a végeredmény függ a rendszer eredeti állapotától és a jól eltalált konfigurációtól.
Amennyiben kellően nagy mennyiségű erőforrás áll rendelkezésre, akkor a HPA képes lenne a lokálisan optimális döntések által mindig növelve a szűk keresztmetszetet, egy globálisan is elfogadható állapotba kerülni.

A szolgáltatáshálók és a skálázó esetében megfigyelt limitációkra több megoldási javaslatot is adtam, melyek a probléma más-más aspektusát megragadva próbálják megszüntetni vagy csökkenteni azt.
Ezen javaslatoknál bemutatásra került az adott eszköz előnye és hátránya is, hogy megfontoltabb döntési alapot kínáljon.

%----------------------------------------------------------------------------
\section{Dolgozatban nem vizsgált kérdések}
%----------------------------------------------------------------------------
A diplomamunka keretén belül sikerült elmélyülni a feladatban, miközben folyamatosan bontakozott ki a terület komplexitása.
Sok feladatot sikerült megvalósítani és a legtöbb vizsgált kérdésre sikerült választ is kapni azonban a tanulmányok során tetemes számban kerültek elő olyan kérdések, amik a rendelkezésre álló idő miatt nem sikerült felderíteni.
Ezen kérdéskörök további vizsgálatot és kutatást igényelnek.

Az elvégzett mérések során bizonyos egyszerűsítésekkel éltem, hogy értelmezhető maradjon a kapott eredmény.
Ilyen egyszerűsítés például az a megkötés, hogy a beérkező kérések száma egyenletes az időben, azonban ez nem minden rendszer esetén van így.
Jól megfigyelhető periódusok jelentkezhetnek egy napon, hónapon vagy éven belül is.
Több projekt is foglalkozik ezzel a jelenséggel és próbálják szintén keresni az erőforrások ideális használatát a historikus adatok elemzésével.
Létezik egy fejlesztés, ami a horizontális pod skálázót szeretné ilyen irányba továbbfejleszteni\citep{predictiveHPAGithub}.
A skálázó segítségével szintén érdemes lenne méréseket futtatni, azonban ennek előfeltétele, hogy dinamikusabban lehessen forgalmat generálni.
Például az időben változóan illetve ezeket több alkalmazásvégpont irányába küldeni.

A dolgozatban elkészített mérések mögötti szolgáltatás hálók a leggyakoribb, egyszerű eseteket fedték le.
Egy valódi mikroszolgáltatás architektúra ennél nagyságrendekkel több komponenssel rendelkezik, melyek közti kapcsolatok is összetettebbek.
Jelenlegi keretrendszert tovább lehetne fejleszteni, hogy a komponensekhez érkező kérések megadott arányok mellett kerüljenek egyik- vagy másik további komponenshez továbbküldésre.
Ezzel kicsit dinamikusabbá lehetne tenni a mostani rendszert, ami minden beérkező kérés esetén azonos kérés-válasz folyamatokat fogja elindítani.

További vizsgálandó kérdés, hogy a kapott eredmények más környezetben hogyan változnak meg.
Például, más eredmény jöhet ki, ha több csomóponttal, nagyobb terhelés mellett, nagyobb erőforrás felhasználással történnek a tesztek.
Sajnos az ilyen klaszterek bérlése drága különösen, hogy a méréseink szándékosan magas mennyiségű processzort használnak, ami alapja szokott lenni a bérelt infrastruktúra utáni számlázásnak.


\subsection{Keretrendszer további használata}
%----------------------------------------------------------------------------
Az elvégzett munka jelentős részét kitevő keretrendszer lehetőséget biztosít további felhasználásra is.
A diplomamunkán belül a feladata az volt, hogy fiktív paraméterekkel rendelkező szolgáltatás hálókat tudjunk elindítani és létrehozni.
Ezen kívül egy igazán hasznos felhasználási mód lehet, ha már létező, üzemelő hálózatokat szeretnénk klónozni.
Tehát létre tudunk hozni kisebb másolatokat az eredeti rendszerről, ami közel azonos kiszolgálási és erőforrás felhasználási paraméterekkel fog rendelkezni, mint a valódi.
Ezután lehetőségünk van fiktív fejlesztéseket végezni az egyes komponensek átkonfigurálásával és az így kapott rendszert tesztelni.
Ezáltal könnyebben lehet megalapozott döntéseket hozni, hogy az aktuális környezetben melyik szolgáltatást érdemes fejleszteni, melyikkel lehet érdemben befolyásolni az eredő nyereséget.

Továbbá az előzőleg bemutatott példán keresztül ki tudjuk próbálni az aktuális rendszerünket egy új környezetben is.
Mindezt úgy tudjuk megtenni, hogy az alkalmazás egységeket alkotó képfájlok mozgatása nem szükséges, ezáltal nem csak gyorsabb lesz a tesztelés, hanem nem kell újabb felek számára elérhetővé tenni a megírt állományokat.
Az így létrehozott mérések betekintést tudnak nyújtani, hogy egy esetleges infrastruktúra váltás milyen változásokkal jár a kiszolgálási paramétereket illetően.