%----------------------------------------------------------------------------
\chapter{Összefoglalás}
\label{sec:summary}
%----------------------------------------------------------------------------


%----------------------------------------------------------------------------
\section{Elvégzett munka}
%----------------------------------------------------------------------------

% % A félév végére sikerült összeállítani egy olyan keretrendszert, amivel előre leírt konfigurációs
% % fájlból kiindulva Kubernetes felett hoz létre egy szolgáltatás-hálót. Az elkészült
% % szolgáltatáshálóba különböző mennyiségű felhasználói forgalmat tudunk generálni. 
% Kiszolgálás közben folyamatosan monitorozzuk a rendszer terheltségét és egyéb metrikáit. 
% A feladat megoldásához tovább kellett fejleszteni a korábbi operátort és webalkalmazásomat.
% Ezen felül Python kódokat írtam, ami képes automatikusan elvégezni a méréseket és az eredményeket vizuálisan megjeleníteni.

% % A rendszer működésének teszteléséhez igényeltem virtuális gépeket és azok telepítése után összeállítottam egy Kubernetes klasztert.
% A klaszter telepítését a \textit{Kubeadm} segítségével végeztem, de kipróbáltam a \textit{Kubespray} telepítőt is. 

% % Az elkészült klaszterben próbaméréseket végeztem, amik sikeresen lefutottak, ezzel bizonyosságot nyert, hogy az egyes elemek nem csak a lokális környezetben képesek együttműködni, hanem a valósághoz közelálló rendszerben is.
% Az elkészült mérési eredményekből  sikerült kézzelfogható grafikont készíteni. 

% % A projekt során úgy érzem sikerült a korábbinál mélyebben megismerni a Kubernetes platformot és ezzel egyidőben belátni, hogy olyan mélységei vannak, hogy mindig lesz újabb és újabb terület amit meg lehet ismerni.
Ráadásul folyamatosan fejlődnek a keretrendszerben felhasznált egységek, így folyamatosan lehetőség van bővíteni is azokat.

Alapvetően jól működik a beépített skálázó.
Pár extra lépéssel javítható lenne.
Vagy nagyon komplex lehetőség vezetne csak sikerre.


%----------------------------------------------------------------------------
\section{Dolgozatban nem vizsgált kérdések}
%----------------------------------------------------------------------------
Megítélésem szerint a két félévben egyenletesen sikerült haladni, amivel egyidejűleg folyamatosan bontakozott ki a feladat komplexitása.
Sok feladatot sikerült megvalósítani és a legtöbb vizsgált kérdésre sikerült választ is kapni.
Ettől függetlenül a tanulmányok során tetemes számban előkerültek olyan kérdések, amik a rendelkezésre álló idő miatt nem sikerült felderíteni.
Ezen kérdéskörök további vizsgálatot és kutatást igényelnének.

Az elvégzett mérések során bizonyos egyszerűsítésekkel éltem, hogy értelmezhető maradjon a kapott eredmény.
Ilyen egyszerűsítés például az a megkötés, hogy a beérkező kérések száma egyenletes az időben, azonban ez nem minden rendszer esetén van így.
Jól megfigyelhető periódusok jelentkezhet egy napon, hónapon vagy éven belül is.
Több projekt is foglalkozik ezzel a jelenséggel és próbálják szintén keresni az erőforrások ideális használatát a historikus adatok elemzésével.
Létezik egy fejlesztés, ami a horizontális pod skálázót szeretné ilyen irányba továbbfejleszteni\citep{predictiveHPAGithub}.

A dolgozatban elkészített mérések mögötti szolgáltatás hálók a leggyakoribb, egyszerű eseteket fedték le.
Egy valódi mikroszolgáltatás architektúra ennél nagyságrendekkel több komponenssel rendelkezik, melyek közti kapcsolatok is összetettebbek.
Jelenlegi keretrendszert tovább lehetne fejleszteni, hogy a komponensekhez érkező kérések megadott arányok mellett kerüljenek egyik- vagy másik további komponenshez továbbküldésre.
Ezzel kicsit dinamikusabbá lehetne tenni a mostani rendszert, ami minden beérkező kérés esetén azonos kérés-válasz folyamatokat fogja elindítani.

További vizsgálandó kérdés, hogy a kapott eredmények más környezetben hogyan változnak meg.
Például, más eredmény jöhet ki, ha több csomóponttal, nagyobb terhelés mellett, nagyobb erőforrás felhasználással történnek a tesztek.
Az a gond, hogy az ilyen klaszterek bérlése drága különösen, hogy a méréseink szándékosan a magas számban használják a processzort, ami alapja szokott lenni a bérelt infrastruktúra utáni számlázásnak.


\subsection{Keretrendszer további használata}
%----------------------------------------------------------------------------
Az elvégzett munka jelentős részét kitevő keretrendszer lehetőséget biztosít további felhasználásra is.
A diplomamunkán belül a feladata az volt, hogy fiktív paraméterekkel rendelkező szolgáltatás hálókat tudjunk elindítani és létrehozni.
Ezen kívül egy igazán hasznos felhasználási mód lehet, ha már létező, üzemelő hálózatokat szeretnénk klónozni.
Tehát létre tudunk hozni kisebb másolatokat az eredeti rendszerről, ami közel azonos kiszolgálási és erőforrás felhasználási paraméterekkel fog rendelkezni, mint a valódi.
Ezután lehetőségünk van fiktív fejlesztéseket végezni az egyes komponensek átkonfigurálásával és az így kapott rendszert tesztelni.
Ezáltal könnyebben és megalapozott döntéseket lehet hozni, hogy az aktuális környezetben melyik szolgáltatást érdemes fejleszteni, melyikkel lehet érdemben befolyásolni az eredő nyereséget.

Továbbá az előzőleg bemutatott példán keresztül ki tudjuk próbálni az aktuális rendszerünket egy új környezetben is.
Mindezt úgy tudjuk megtenni, hogy az alkalmazás egységeket alkotó képfájlok mozgatása nem szükséges, ezáltal nem csak gyorsabb lesz a tesztelés, hanem nem kell újabb felek számára elérhetővé tenni a megírt állományokat.