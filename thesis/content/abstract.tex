\pagenumbering{roman}
\setcounter{page}{1}

\selecthungarian

%----------------------------------------------------------------------------
% Abstract in Hungarian
%----------------------------------------------------------------------------
\chapter*{Kivonat}\addcontentsline{toc}{chapter}{Kivonat}

Az elmúlt években megfigyelhető egy jelentős szemléletmódváltás a fejlesztett és használt alkalmazások körében, illetve ezzel összefüggően a futtató környezetben is.
Egyre több informatikai szereplő ismeri fel, hogy milyen előnyöket tud nyújtani a monolitikus alkalmazások cseréje mikroszolgáltatások architektúrára.
Az új trenddel párhuzamosan, illetve egymást erősítve folyamatosan kezd csökkenni az alkalmazásfejlesztő és üzemeltető feladatkörök távolsága.

A frissen készülő alkalmazások egyre nagyobb része direkt konténerizálva, a felhőre optimalizáltan készül.
Sokak számára vonzó alternatívát kínál a könnyű és kölcséghatékony belépésének és a skálázhatóságának köszönhetően.
Az így elkészített alklamazások üzemeltetése is új kihívásokkal szembesül és erre az igényre jelent meg az azóta folyamatosan fejlődő és növekvő Kubernetes konténer orkesztrációs platform.

A Kubernetes platform lehetőséget kínál számunkra több metódus alapján skálázni az adott alkalmazásunkat a beérkező terheléstől függően.
A diplomamunkában azzal a kérdéskörrel foglalkozopruim, hogyan viselkedik a rendszer, ha a rendelkezésre álló erőforrások limitáltak vagy pedig minél költséghatékonyabban szeretnénk üzemeltetni az alkalmazást. 

Keressük a jelenelgi implementációban létező limitációkat, hogy milyen helyezetek adódhatnak, amikor ideálistól eltérő erőforrás elosztás és használat figyelhető meg.

Ehhez a feladathoz készítettem egy mikroszolgáltatás architektúrájú alkalmazás szimulációjához szükséges környezetet. Ezzel lehetőségünk van tetszőleges konfigurációval Kubernetes felett elindíteni egy szimulált környezetet, ahol személyre szabhatjuk a válaszidőket és processzor használatot.
A környezeten előállított alkalmazásokat pedig meg kellett figyelni, hogyan viselkednek a nagy számú külső terhelés hatására.

A mérések során felderített, előnytelen működési esetekre megoldást javaslunk, amivel kiegészítve az alap infrastruktúrát csökkenthető az erőforrások pazarlása, így környezet- és pénztárcabarát megoldás is egyben.

\vfill
\selectenglish


%----------------------------------------------------------------------------
% Abstract in English
%----------------------------------------------------------------------------
\chapter*{Abstract}\addcontentsline{toc}{chapter}{Abstract}

%TODO


\vfill
\selectthesislanguage

\newcounter{romanPage}
\setcounter{romanPage}{\value{page}}
\stepcounter{romanPage}