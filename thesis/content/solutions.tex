%----------------------------------------------------------------------------
\chapter{Megoldási lehetőségek}
\label{sec:solutions}
%----------------------------------------------------------------------------

Ebben a fejezetben szeretném bemutatni, hogy a korábban \aref{sec:results} fejezetben látott mérések alapján milyen megoldási lehetőségek jöhetnek számításba.
Természetesen minden bemutatott megoldásnak megvan a saját erőssége és gyengesége, amiket a következő alfejezetben részletesen is ismertetek.

%----------------------------------------------------------------------------
\section{Feltárt probléma rövid összefoglalása}
%----------------------------------------------------------------------------

%----------------------------------------------------------------------------
\section{Lehetséges eszközök}
%----------------------------------------------------------------------------

\subsection{Readyness probe}
%----------------------------------------------------------------------------

\subsection{Konténer specifikus metrikák alapján}
%----------------------------------------------------------------------------

\subsection{Service Mesh}
%----------------------------------------------------------------------------

\subsection{Gloo Edge}
%----------------------------------------------------------------------------

\subsection{Okos skálázó}
%----------------------------------------------------------------------------

pl: nyomon követi milyen arányban használják az erőforrásokat

Amdahl's törvényt is figyelembe  veszi
VPA és HPA együtt működikN