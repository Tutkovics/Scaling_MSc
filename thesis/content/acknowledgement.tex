%----------------------------------------------------------------------------
\chapter*{\koszonetnyilvanitas}
\addcontentsline{toc}{chapter}{\koszonetnyilvanitas}
%----------------------------------------------------------------------------

% Gábornak: Sok segítség, nem csak a szakmára tanított felkészülni, hanem az életre.  
% Családnak: Pénz és idő
% Schönherz kollégium: egyetemi társaság, sok tanulás, VM-ek


Nehéz szavakba önteni a dolgozat készítése közben feltörő érzéseket.
Egyszerre van jelen a megkönyebbülés, a hála, az alkotás szeretete, a kielégíthetelen tudásvágy és az elmúlás is.
A diplomamunkám nem jöhetett volna létre számos támogatás nélkül, amit a körülöttem lévő emberek áldozatos munkájukból kaptam.
Szeretnék köszönetet mondani, hogy az egyetemen eltöltött időm alatt a legkülönfélébb területeken tudást és élményt tudtam gyűjteni.
Érzem és tudom, hogy ezek által korábbi lényemhez képest több lettem és ezzekkel felvértezve a társadalomnak is hasznosabb tagja lehetek és szívből remélem, hogy többet tudok majd vissza is adni.

Szeretném külön kiemelni Dr. Rétvári Gábor konzulensemet, mivel az egyetemen végzett minden egyes projekttárgyat a felügyelete alatt készíthettem.
Szakmai tanácsaival mindig tovább tudta gurítani a megakadónak tűnő feladatokat és végig iránymutatással segítette, hogy ne vesszek el az egyes részletekben és tudatosította, hogy a nagyképet figyelve jelenleg hol tartok.
Mindezek már önmagukban köszönetet érdemelnek, de nem csak konzulensemmé vált Gábor, hanem a kötelező feladatokon túl mentorommá is.
Számíthattam rá, amikor az egyes feladatok nehézsége és nagysága miatt motivációmat vesztettem illetve egyedülálló személysége már önmagában motiváló.
Megmutatta, hogy egyenességgel, nyitottsággal és örök jókedvvel milyen szélessé nyilhat ki a világ.
Hálás vagyok érte, hogy együtt dolgozhattunk és tanulhattam tőle minden egyes konzultáció vagy találkozás alkalmával.

Köszönettel tartozom a családom támogatásáért, hogy egész életemben fontos szerepet szántak a tanultatásnak.
Tudom, hogy jelentős áldozatokkal járt ez a folyamat de remélem sikerült a legjobban kihasználni az eddigi lehetőségeket és hosszútávon meg fog térülni.
Nekik köszönhetem, hogy támogatásukkal lehetőségem nyílt az egyetemi éveimet a kellő szabadsággal megélni, ezáltal is felfedezve illetve jobban megértve saját magam és a világ működését.

Végül pedig szeretném megemlíteni a Schönhez kollégium közösségét, ahol lehetőségem volt olyan dolgokat is kipróbálni, amik nem  közvetlenül a szakmához, hanem a tágabban értelemben vett élethez hasznos tapasztalatokkal szolgáltak.
Kipróbálhattam magam több csoport élén is, ezzel is tágítva a saját komfortzónám határát és ösztönöztek az alkotás és gondolkodás aktív folyamatára.