%----------------------------------------------------------------------------
\chapter*{\koszonetnyilvanitas}
\addcontentsline{toc}{chapter}{\koszonetnyilvanitas}
%----------------------------------------------------------------------------

Ez nem kötelező, akár törölhető is. Ha a szerző szükségét érzi, itt lehet köszönetet nyilvánítani azoknak, akik hozzájárultak munkájukkal ahhoz, hogy a hallgató a szakdolgozatban vagy diplomamunkában leírt feladatokat sikeresen elvégezze. A konzulensnek való köszönetnyilvánítás sem kötelező, a konzulensnek hivatalosan is dolga, hogy a hallgatót konzultálja.

% Gábornak: Sok segítség, nem csak a szakmára tanított felkészülni, hanem az életre.  
% Családnak: Pénz és idő
% Schönherz kollégium: egyetemi társaság, sok tanulás, VM-ek


Nehéz szavakba önteni a dolgozat készítése közben feltörő érzéseket.
Egyszerre van jelen a megkönyebbülés, a hála, az alkotás szeretete, a kielégíthetelen tudásvágy és az elmúlás is.

Szeretnék köszönetet mondani, hogy az egyetemen eltöltött időm alatt a legkülönfélébb területeken tudást és élményt tudtam gyűjteni.
Tudom és érzem, hogy ezek által korábbi lényemhez képest több lettem és ezzekkel felvértezve a társadalomnak is hasznosabb tagja lehetek és többet tudok majd vissza is adni.
Ez azonban nem a saját érdemem, hanem ehhez nagyon sokan hozzájárultak.
A következő sorokban szeretném külön kiemelni, az ebben legnagyobb szerepeket játszó személyeket.

Szeretném kezdeni a konzulensemmel. 