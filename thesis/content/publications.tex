%----------------------------------------------------------------------------
%\chapter{\Kubernetes}
%\label{sec:Kubernetes}
%----------------------------------------------------------------------------
\chapter{Irodalomkutatás}
\label{sec:Publications}
A félév során több publikációt is elolvastam, hogy tisztább legyen a kutatási terület. A következőben szeretném összefoglalni, hogy eddig milyen irányban történtek haladások.

Több oldalról meg lehet fogni a skálázás és szorosan hozzá köthető erőforrás elosztás területét. Minden kutatás kicsit más szempontból vizsgálja, más áll a középpontban.

Vannak: reaktív és prediktív modellek.
Prediktív --> ML, MC
  - a nap / hónap / év ciklikusságát használja 
reaktív: fix szabélyrendszer (pl: 50\% kihasználtság felett --> skálázás!)



