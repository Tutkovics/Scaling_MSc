%----------------------------------------------------------------------------
\chapter{\bevezetes}
%----------------------------------------------------------------------------

%----------------------------------------------------------------------------
\section{Motiváció}
%----------------------------------------------------------------------------

Nagyobb távlatól tekintve az informatikai rendszerekre számos tendenciát figyelhetünk meg a technológia változásával.
Ilyen folyamat, hogy régebben jellemzően dedikált csapat, dedikált nyelven, dedikált alkalmazást, dedikált hardveres erőforrásra fejlesztett éveken keresztül. 
Ennek az eredménye, a mostanra megbonthatatlanul nagyra nőtt rendszerek, amelyeknek külön szervereket kell biztosítani, hogy azok hiba nélkül tudják kiszolgálni a beérkező igényeket.
Több területen használnak még ilyen rendszert, mely megbízható de továbbfejlesztése bonyolult és emiatt költséges is.

Ezután következett, a realizáció, hogy nem éri meg minden ilyen szolgáltatáshoz egy külön fizikai szervert fenntartani, hiszen a kihasználtsága nem feltétlen indokolja, és a drágán megvásárolt erőforrások feleslegesen állnak.
Ehhez plusz egy katalizátor volt a folyamatosan megjelenő virtualizációs technológiák, amik lehetővé tették, hogy a fizikai rendszert absztrahálva, különböző virtuális környezetet hozzunk létre azok felett.
A megoldásnak köszönhetően lehetőség nyílt előre definiált lépésekkel azonos tulajdonsággal bíró rendszereket létrehozni igény szerint. 
Természetesen mindezt úgy, hogy a fizikailag rendelkezésre álló erőforrások kihasználtsága is javult.

A korábban vázolt lehetőséggel mai napig sokan élnek, azonban a fizikai életben is megjelenő rugalmasság és agilitás iránti vágy köszön vissza az informatikában is. 
Aki nem tud lépést tartani a gyors fejlődésben az rövid időn belül lemarad és hátrányba kerül a versenytársaihoz képest. 
Az érem másik oldala viszont az, hogy aki időben észreveszi a lehetőséget, az hirtelen nagy előnyt tud szerezni. 
Erre jó példa az Amazon\citep{amazon} és a Netflix\citep{netflix} is.  

Egyre szélesebb körben terjednek el a konténerizációs technológiák és vele együtt az úgynevezett mikro-szolgáltatások.
A paradigma értelmében a korábbi nagyobb kódegységeket szét lehet bontani több kisebb kódbázisra, ami számtalan előnyt jelent az alkalmazás fejlesztéséhez.
Mivel a kisebb egységeket könnyebb megérteni mint a teljes kódbázist, az új fejlesztők számára könnyebb becsatlakozni fejlesztésbe.
A kisebb elemekre bontott kód lehetőséget biztosít arra is, hogy az egyes komponenseket több
nyelven és többfajta keretrendszerben fejlesszük, minden komponens számára az optimális fejlesztési környezetet kiválasztva. 
Végül, ha egy komponenst szeretnénk teljesen elölről újraírni, akkor ez a rendszer többi részét nem érinti.
Ezek által lehetőség nyílik egy agilis rendszer kialakítására.

A végbemenő tendenciaváltás egyik nagy szereplője a Kubernetes rendszer, ami lehetőséget biztosít a konténerizációs technológia széles körű és egyszerű használatára.

%----------------------------------------------------------------------------
\section{Feladat meghatározása}
%----------------------------------------------------------------------------
A korábban említett paradigma váltással lépést kell tartani az alkalmazás üzemeltetésekor is. 
Fontos szempont az elkészült, mikro-szolgáltatásokból felépülő hálózat skálázásának kérdése. 
Sokan választották az új megközelítést, mert ígérete szerint könnyen képes a beérkező kérések okozta terheléssel arányosan skálázódni.
Ez egy triviális feladat egészen addig amíg az egyes egységeket különálló, másokkal nem összefüggő részként kezeljük.
Ezt megtehetjük és látszólag legtöbben meg is teszik komolyabb fennakadások nélkül. 
Azonban a valóság ennél jóval komplexebb. Figyelembe kell venni, hogyan kapcsolódnak egymáshoz a mikro-szolgáltatások, azok milyen és mennyi erőforrást használnak, mi történik a beérkező kérésekkel.
Látható, hogy pár paramétert bevéve az egyenletbe már nem is könnyű feladat. 
Különösen megnehezíti a feladatot, ha a rendelkezésre álló erőforrásokat (például: processzor, memória, hálózat, I/O) globálisan optimálisan szeretnénk elosztani, hogy a lehető legjobb kiszolgálást tudja biztosítani a rendszer. 

A diplomatervezés során ezt a kérdéskört kell megismernem és a felmerülő kérdésekre választ keresni. 
Első lépésben szeretném bemutatni a használt környezetet, a Kubernetes felépítését illetve, hogy
milyen skálázási megoldások léteznek és melyiket hogyan támogatja.
% Első lépésként fontos tisztában lenni a használt platformmal, a Kubernetessel megnézni benne milyen skálázási megoldások vannak jelenleg és melyik hogyan van támogatva. 
Konkrét példán keresztül bemutatom a HPA megoldását. 

Egy egyedi keretrendszert kellett összeállítani, ami képes tetszőleges szolgáltatáshálót létrehozni
Kubernetesen belül és azt lekérdezésekkel terhelni. Ezáltal lehetőségünk nyílik szabadon
konfigurálható körülmények között vizsgálni a Kubernetes beépített skálázóját, illetve az
irodalomban javasolt egyéb megoldásokat is. Az így nyert eredményekből következtetéseket tudunk
levonni az egyes implementációk dinamikáját, költségeit, hatékonyságát illetően.

A feladat része a kapott mérési eredmények vizualizációja is, így könnyebben láthatóvá válnak a
skálázási megoldások és a megoldások közötti különbségek is. 

Amennyiben a mérési eredmények alapján fejleszthető lenne egy új stratégia, akkor azt implementálás
után szintén meg lehet vizsgálni és kiértékelni a hipotézist.

%A beépített skálázás értékeléséhez méréseket kell végezni minél több szolgáltatáshálón, hogy látni lehessen melyik rendszer hogyan viselkedik a terhelés függvényében. 
%Az adatok gyűjtéséhez kell egy külön rendszer, ami képes lesz tetszőleges tulajdonságokkal rendelkező szolgáltatáshálót létrehozni és azokon szimulációkat végezni.

%Az így gyűjtött eredményeket ki kell értékelni, hogy jobban megértsük a rendszert és következtetéseket tudjunk levonni a skálázás minőségét illetően. 
%Amennyiben azt találjuk, hogy bizonyos lépésekkel optimálisabb skálázás valósítható meg, akkor erre
%egy példát kell mutatni és bizonyítani a hipotézis igazát. 

%----------------------------------------------------------------------------
\section{A dolgozat felépítése}
%----------------------------------------------------------------------------
A dolgozat fejezetei úgy lettek sorba állítva, hogy azok a lehető legérthetőbb módon mutassák be a kapott eredményeket. Ehhez viszont az általános részektől kell indulni és így mutatva be az egyre konkrétabb részeket. 
A \ref{sec:Kubernetes}. fejezetben szó lesz a Kubernetes architektúrájáról, illetve hogyan támogatja beépített módon az automatikus skálázást. 
A \ref{sec:Publications}. fejezetben bemutatom, hogy milyen kutatások készültek ebben a témában, ki mire helyezte a hangsúlyt és hova sikerült eljutni.
Ezután a \ref{sec:system} bemutatom az általam készített keretrendszert, annak építő elemeit a megvalósítás mérnöki döntéseit. 
Az \ref{sec:results}. fejezetben lesz szó az elvégzett mérésekről, illetve a kapott eredmények elemzéséről.
A mérési eredmények alapján pár megoldási javaslatot teszek \aref{sec:solutions}. fejezetben, vázolva az előnyeiket és hátrányaikat. 
Végül \aref{sec:summary}. fejezetben összefoglalom a diplomamunkám eredményeit és kitérek az újonnan felmerült kérdésekre.
