%----------------------------------------------------------------------------
\chapter{\bevezetes}
%----------------------------------------------------------------------------

Mi legyen benne:
A bevezető tartalmazza a diplomaterv-kiírás elemzését, történelmi előzményeit, a feladat indokoltságát (a motiváció leírását), az eddigi megoldásokat, és ennek tükrében a hallgató megoldásának összefoglalását.

Tendencia: bare-metal --> virtualizáció --> konténerizáció --> felhő --> új alkalmazás architektúra (SM) 

Nagyobb távlatól rátekintve az informatikai rendszerekre számos tendenciát figyelhetünk meg a technológia változásával. Ilyen folyamat, hogy régebben jellemzően dedikált csapat, dedikált nyelven, dedikált alkalmazást, dedikált hardveres erőforrásra fejlesztett éveken keresztül. Ennek az eredménye, a mostani megbonthatatlnaul nagy rendszerek, amelyeknek külön szervereket kell biztosítani, hogy azok hiba nélkül tudják kiszolgálni a beérkező igényeket. Több területen használnak még ilyen rendszert, mely megbízható de továbbfejlesztése bonyolúlt és emiatt költséges is.

Ezután következett, a realizáció, hogy nem éri meg minden ilyen szolgáltatáshoz egy külön fizikai szervert fenttartani, hiszen a kihasználtsága nem feltétlen indokolja, és a drágán megvásárolt erőforrások feleslegesen állnak. Ehhez plusz egy katalizátor volt a folyamatosan megjelenő virtualizációs technológiák, amik lehetővé tették, hogy a fizikai rendszert absztrahálva, különböző virtuális környezetet hozzunk létre azok felett. A megoldásnak köszönhetően lehetőség nyílt előre definiált lépésekkel azonos tulajdonsággal bíró rendszereket létrehozni igény szerint. Természetesen mindezt úgy, hogy a fizikailag rendelkezésre álló erőforrások kihasználtsága is javult.

A korábban vázolt lehetőséggel mai napig sokan élnek, azonban a fizikai életben is megjelenő rugalmasság és agilitás iránti vágy köszön vissza az informatikában is. Aki nem tud lépést tartani a gyors fejlődésben az rövid időn belül lemarad és hátrányba kerül a versenytársaihoz képest. Az érem másik oldala viszont az, hogy aki időben észreveszi a lehetőséget, az hirtelen nagy előnyt tud szerezni [amazon/netflix].

Egyre szélesebb körben terjednek el a konténerizációs technológiák és vele együtt az úgynevezett mikro-szolgáltatások. A paradigma értelmében a korábbi nagyobb kódegységeket szét lehet bontani több kisebb kódbázisra, ami így számtalan előnyt jelent az alkalmazás fejlesztéséhez. A kisebb egységeket könnyebb megérteni, így könnyebb új tagoknak becsatlakozni a fejlesztésbe, nem kell rögtön egyben látnia a rendszert. Valamint lehetőséget biztosít, hogy több nyelven, keretrendszerben legyenek megoldva az alfeladatok. Ezáltal minden lépéshez a számára legelőnyesebb környezetet lehet megválasztani. Valamint ha egy komponenst szeretnénk teljesen előlről fejleszteni, akkor az a rendszer többi részét nem érinti. Ezek által lehetőség nyílik egy agilis rendszer kialakítására.

A konténerizációs technológiák másik nagy előnye, hogy környezettől független módon lehet 


A következő fejezetekben szeretném ismertetni a diplomaterv elvégzéséhez szükséges feladatokat. Az x. fejezet foglakozik stb.

%----------------------------------------------------------------------------
\section{A dolgozat felépítésee}
%----------------------------------------------------------------------------
A következő fejezetekben bemutatom az általam használt rendszerek alapvető feléptését illetve  azok számunkra fontos részeit. A \ref{sec:Kubernetes} fejezetben szó lesz a Kubernetes részeiről, azok kapcsolatáról, illetve hogyan támogatja beépített módon az automatikus skálázást. Ezután az általunk írt
